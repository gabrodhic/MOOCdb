\section{Discussion}
The initial results show the Louvain Method obtains a higher modularity than spectral clustering in two offerings of 6.002x. In the third offering, only the Louvain Method produced results. The Louvain Method also achieves this maximum modularity with a lower number of of communities when compared to spectral clustering.

Spectral clustering requires the number of communities to be specified. In order to optimize modularity, CommunityFinder does a brute force search over the number of communities. Figure \ref{modularity} shows how the modularity varies as we vary the number of communities. We can see that increasing the number of communities increase modularity up to a certain point. After that, further increases in number of communities decreases the modularity. 

The other important consideration is performance. The Louvain Method vary clearly outperforms spectral clustering on our dataset. The running time of Louvain method for 6.002x Fall 2012 is nearly 10x faster than using spectral clustering to determine 100 communities on equivalent Interaction Graph. The advantage becomes clearer when we determine the optimal number of communities in spectral clustering because we must actually run the the algorithm multiple times. For example, in our testing, we tested nearly 700 different values for the number of communities in 6.002x Fall 2012. This further amplifies the performance advantage of the Louvain Method.

Overall, these two results suggest that the Louvian Method is better for community detection under the goal of modularity optimization. If we consider another metric the results may change. Additionally, we only ran this analysis on three datasets, all from offerings the same course. We may see different results for other courses, especially ones in disciplines other than computer science.