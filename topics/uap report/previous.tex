\section{Previous Work}
The problem of identifying communities is very hard and not yet satisfactorily solved. The dependency on domain makes it difficult to generate general purpose algorithms. Further, some domains such as social network data require not only accurate, but efficient algorithms to process large amounts of data. However, there is a large interdisciplinary effort to solve it because of the number of applications.

The community structure in a network of interactions represented as a graph where entities are vertices and interactions are edges is a clustering of a graph such that vertices in a cluster have more edges to vertices in the same cluster and fewer edges to vertices in other clusters\cite{2010PhR...486...75F}. How we interpret the clusters that form is often dependent on the domain of the problem, but they often are interpreted as vertices playing a similar role. For example, given an network of how scientific researchers have collaborated with each other, we would expect to find that communities correspond to the different scientific disciplines. 

In the domain of online education, we might want to identify the students that take various roles in a class. Huang et al. \cite{Huang:2014:SBM:2556325.2566249} focused on the most focal subset of contributors on MOOC forum that they label as \emph{superposters}. They suggest the \emph{superposters} can be models for the ideal student because they often make high quality posts. Their study seeks to examines contribution patterns, demographics, and course performance and enrollment of these \emph{superposters}. The researchers conclude that these posters make high-value contributions and also encourage further student engagement. 

Other researchers analyzed post content from business strategy class on Coursera \cite{DBLP:journals/corr/GillaniEOHR14}. This meant considering over 15,600 posts. They looked at 5 aspects a student's postings and then used Bayesian Non-negative Matrix Factorization to extract communities. They looked specifically at two sub-forums and identified communities such as committed crowd engagers, discussion initiators, strategists, and individuals.
